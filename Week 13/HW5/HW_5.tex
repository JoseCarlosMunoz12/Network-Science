\documentclass[12pt,english]{article}
\usepackage[a4paper,bindingoffset=0.2in,%T
            left=1in,right=1in,top=1in,bottom=1in,%
            footskip=.25in]{geometry}
\usepackage{blindtext}
\usepackage{caption}
\usepackage{subcaption}
\usepackage{titling}
\usepackage{amssymb}
\usepackage{amsmath}
\usepackage{listings}
\usepackage{lettrine} 
\usepackage{tikz}  
\usepackage{color}
\setlength{\parskip}{12pt}
\begin{document}
\newgeometry{left=0.8in,right=0.8in,top=1in,bottom=1in}
\begin{center}
    \Large
    \textbf{Homework 5}\\
    \small
    \today\\
    \large
    Jose Carlos Munoz
\end{center}
%===============================
\section*{1}
1) The nodes that will have the larget cascade are node 8, 2 and 3. node 8 will cascade into 6,7 on the first iteration and then into 5 in the second iteration. Node 2 will cascade into 1 on the first iteration, 3 on the second and then on 4 for the final iteration. Node 3 will first cascade into 1, then into 2, and finally into 4. All three of theses starting nodes will have a cascade of 4.\\
2)There are two possible sets of initial adopters that will cause a complete cascasde. the first set is node 8 and node 2. On the first iteration nodes 6,7 and 1 will change. The second iteration wil have node 5 and 3 converted. And the final node 4 will be converted.\\
The second set will also have a complete cascade is node 3 and node 8. The first iteration node 6,7 and 1 will be converted. Then the second iteration node 5 and node 2 will convert. And node 4 will convert in the final iteration.
\section*{2}
For a cascade to occur, the fractional of neighbors must be higher than the value q. In this situation,all the q are the same value. To have a complete cascade, we have to find the node with the highest amount of neighbors. In this situation it is node 1. Node 1 has a total of 5 neighbors. The threshold should be at $\frac{1}{5}$. So the maximum threshold for a complete cascade should be $\frac{1}{5}$ or at 20\%. When this is applied, the neighbors of 5 will convert (nodes 5,4,2 and 3). Then the next iteration nodes 6,8, and 9 will convert. Leading it to be a complete cascade.
\section*{3}
1) The influence of A is 4. This is because that in the first iteration node 2 is activated. Then in the next iteration, node 4 is activated. There is no more nodes that can be activated, so the influence of A is 4\\
2)The set of node 6, 3 and 5 have a influence of 6. This is because in the first iteration node 2 id activated. Then the next node to be acitavated is 4. And then node 1 is finally activated. There will be no more nodes to be activated. So the total influence of this set A $\{3,5,6\}$ is 6.
\section*{4}
1) In this situation, the transmition rate is 0.01* 10 or 0.1 and the recovery rate is 0.4. Since the transmission rate is lower than the recovery rate, the infection will shrink.\\
2)In this situation, the transmition rate is 0.03 * 10 or 0.3 and the recovery rate is 0,1. Since the transmission rate is higher than the recovery rate, the infection will spread.\\
3)In this situattion, the transmission rate is 0.3 * 10 or 0.3 and the recovery rate is 0.2. Since the transmission rate is higher than the recovery rate, teh infeciton will spread.
\section*{5}
1) in a weeks time, an infected person would have seen a total of 70 possible contancts. From these contacts there is a 0.03 chance of a person of becoming infected. So our Infection rate is $\beta = 0.03 * 70$ or 2.1. Sinec our infection rate is above 1, the infection in this situtation will spread.\\
2) In this scenerio, the infected person has the same total amount of contactt in a weeks time, 70 people. The odds of being infected is 0.02 So our Infection rate is $\beta = 0.02 * 70$ or 1.4. Here our infection rate is above 1, so the infection will spread in this situation.\\
3)In this scenerio, the same amount of people are contacted and the chance of being infected is 0.01. So our Infection rate is $\beta = 0.01 * 70$ or 0.7. Since the infection rate is below 1, the infection will not spread but shrink.
\end{document}