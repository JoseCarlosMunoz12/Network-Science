\documentclass[12pt,english]{article}
\usepackage[a4paper,bindingoffset=0.2in,%T
            left=1in,right=1in,top=1in,bottom=1in,%
            footskip=.25in]{geometry}
\usepackage{blindtext}
\usepackage{titling}
\usepackage{amssymb}
\usepackage{amsmath}
\usepackage{listings}
\usepackage{lettrine} 
\usepackage{tikz}  
\usepackage{color} 
 \usetikzlibrary{shapes, arrows, calc, arrows.meta, fit, positioning} % these are the parameters passed to the library to create the node graphs  
\tikzset{  
    -Latex,auto,node distance =0.6 cm and 1.3 cm, thick,% node distance is the distance between one node to other, where 1.5cm is the length of the edge between the nodes  
    state/.style ={ellipse, draw, minimum width = 0.9 cm}, % the minimum width is the width of the ellipse, which is the size of the shape of vertex in the node graph  
    point/.style = {circle, draw, inner sep=0.18cm, fill, node contents={}},  
    bidirected/.style={Latex-Latex,dashed}, % it is the edge having two directions  
    el/.style = {inner sep=2.5pt, align=right, sloped}  
}  
\setlength{\parskip}{12pt}
\begin{document}
\newgeometry{left=0.8in,right=0.8in,top=1in,bottom=1in}
\begin{center}
    \Large
    \textbf{Homework 1}\\
    \small
    \today\\
    \large
    Jose Carlos Munoz
\end{center}
%===============================
\section*{1}
1) The maximum number of edges that a simple graph can have is $\frac{n(n-1)}{2}$\\
2) The maximum number of edges that a simple graph can have is $(n-1)$\\
\section*{2}
This graph is not a stronly connected graph. This is because Vertex B and E does not have a path that directly leads to them. 
\section*{3}
This simple graph is not a bipartite graph. This is beause it is not possible to generate two subset graphs that within the same set are adjacent
\section*{4}
from the graph we can see that node 1, 2, 6 and 7 are symmetric to each other. Node 3 and 5 are also symetric.
\begin{align*}
C_{1,2,6,7}& =\frac{7}{0 +1 +1 +2 +3 +4 + 4} & & C_{3,5} & =\frac{7}{1 +1 +0 +1 +2 +3 + 3} & &C_4 & =\frac{7}{2 + 2 + 1 + 0 + 1 + 2 + 2}\\
C_{1,2,6,7}& =\frac{7}{15}                             & & C_{3,5} &=\frac{7}{11}                              & &C_4 & =\frac{7}{10}
\end{align*}
so we can say that the Closeness of nodes 1,2,6,7 are $\frac{7}{15}$, nodes 3 and 5 are $\frac{7}{11}$ and node 4 is $\frac{7}{10}$
\section*{5}
Just by looking at the graph, we can tell that the betweeness for Node 1,3,4,5 are zero.
\begin{align*}
B_{2}& =\frac{1}{{5-1\choose 2}} * (\frac{0}{1}+\frac{1}{1}+\frac{1}{1}+\frac{1}{1}+\frac{1}{1}+\frac{1}{1})\\
B_{2}& =\frac{1}{{4\choose 2}} * (0 + 1+1+1+1+1)\\
B_{2} & =\frac{1}{6} * (5)\\
B_{2} & =\frac{5}{6}
\end{align*}
So the Betweeness of nodes 1,3,4,5 are zero and the betweeness of node 2 is $\frac{5}{6}$
\section*{6}
1)\\
From thie graph, we know that vertices 6 and 7 are symmetric. So both will have the same closeness factor
\begin{align*}
C_1 &= \frac{7}{0 + 1 + 1 +2 +2 +3 +3}=\frac{7}{12}\\
C_2 &= \frac{7}{1 + 0 + 1 +2 +1 +2 +2}=\frac{7}{9}\\
C_3 &= \frac{7}{1 + 1 + 0 +1 +2 +3 +3}=\frac{7}{11}\\
C_4 &= \frac{7}{2 + 2 + 1 +0 +3 +4 +4}=\frac{7}{16}\\
C_5 &= \frac{7}{2 + 1 + 2 +3 +0 +1 +1}=\frac{7}{10}\\
C_{6,7} &= \frac{7}{3 + 2 + 3 +4 +1+2 +0}=\frac{7}{15}
\end{align*}
Vertex 2 is the one with the higest Closeness
2)\\
All combination of paths have a total of 1 shoretst path. So fro $P(u,v)$ in which u and v do not equal each other is 1. From the Graph we can see that for vertices 2,3 and 5, thier betweeness is non zero. While the rest are 0.
\begin{align*}
B_2 &= \frac{1}{{7-1\choose 2}}*9=\frac{9}{15}\\
B_3 &= \frac{1}{{7-1\choose 2}}*4=\frac{4}{15}\\
B_5 &= \frac{1}{{7-1\choose 2}}*8=\frac{8}{15}\\
\end{align*}
So the vertex with the largest betweenness is vertex 2
\end{document}